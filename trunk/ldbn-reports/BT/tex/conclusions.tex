\chapter{Conclusions}
\label{chap:conclusion}
As the core of a previous thesis~\cite{mt1}
a Web-based environment for learning normalization of relational database schemata
called LDBN 1.0 (Learn DataBase Normalization) was developed to
enhance teaching and learning of relational-database normalization. 
In this thesis we developed some crucial extensions to LDBN.

The main design goal of this project was to develop an extension of the system which is capable of 
visualizing FDs. The visualization approach is based
upon templates found in popular textbooks such as~\cite{bdb1} and~\cite{bdb2}. Secondary goals of the
thesis was to increase the usability of the system by: 

\begin{enumerate}
	\item Reducing the number of unnecessary assignments
	submitted to the system.
	\item Emphasizing assignments submitted by database course lecturers.
	\item Giving more control to lecturers over the system.
\end{enumerate}

Another major goal was the
development of a user-friendly, fast and most importantly robust
user interface (UI) for the new features of the system. 
Indeed, the UI is one of the key features of LDBN 1.1 and critical for its
success. Therefore, during the implementation process 
most of our efforts were concentrated on the development and on the improvement of the UI. 
We believe that it will be well received by both students and lecturers.
Furthermore, we hope that advanced features such as drag and drop will 
increase the usability of the environment.

It should be mentioned that LDBN is developed as an open source project 
under the Apache License, Version 2.0~\cite{walv20}. 
Source code and documentation are available at the project Web page~\cite{wldbnpp}. 
The development of LDBN will continue, and we
hope that soon a community will be
built around the project, and that it will attract other developers as well. 
Possible directions for improving LDBN 1.1 are presented in the following section. 

\section{Limitations and Future Work}
There are many other different approaches for visualizing FDs.
A logical next step for LDBN 1.1 would be to support
even more types of visualization diagrams. 
However, this was not a requirement for the thesis and we believe the provided 
visualization is sufficient at the moment. 

Another possible direction for future development is support for printing the visualization output.
However, this is not possible at the moment as it would require rendering the visualization on the server side as
a static image file and sending it back to the client. 
This is complicated by the fact that our visualization relies heavily on
the \verb=<canvas>= HTML5-element which is only available on the client. 
A rendering on the server side would require a totally different approach. 
Support for printing was more of a desire than a requirement and was therefore not implemented 
in the final version of LDBN, on the other hand, the user can still print the visualization output
by using screen capture.
