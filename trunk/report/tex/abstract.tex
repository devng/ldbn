Database normalization is a technique for designing relational database tables 
to minimize duplication of information in order to safeguard the database 
against certain types of logical or structural problems, namely data anomalies. 
Therefore database normalization is a central topic in database theory, and its 
correct understanding is crucial for students. Unfortunately  the subject it is 
often considered to be dry and purely theoretical and it is widely being disregarded 
by the students. A web-based learning environment is developed to give 
students an interactive hands-on experience in database normalization process. It
also provides lecturers with an easy way for creating and testing assignments 
on the subject.  
The learning environment is suitable for relational database and design and data 
management courses. 

This report describes the design and development of LDBN 
(Learn DataBase Normalization) - a reference implementation of the learning environment.
It also discuss problems that lie within educational and web-based software development.
A tutorial on the right usage of LDBN is also provided.   
%keywords (do we need them?)
%\newline
%\newline
%\textbf{Keywords:} Database Normalization, Relational Data Model, Functional Dependency, 
%Third Normal Form, Boyce-Codd Normal Form, Embedded Functional Dependencies
