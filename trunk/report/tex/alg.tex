\section{Normalization Algorithms}
\label{sec:alg}
In this section we present different normalization algorithms which are used
in LDBN. A proof of their correctness and complexity would be beyond the 
scope of this report and therefore it is omitted. We provide the
reader with textual description and/or pseudocode for each algorithm. 
A reference to additional information is also given. However, in this section we do not
illustrate how the algorithms are applied in the learning environment in order
to perform the \textit{Solve Assignment} and the \textit{Check Solution} functions.
This is done in Section~\ref{sec:keyfunctions}.

We can divide the normalization algorithms used by LDBN into two different groups:

\begin{enumerate}
  \item \textit{Algorithms for testing} are used to test whether a relation is satisfying a certain normal form criteria.
  \item \textit{Decomposition algorithms} are used to automatically decompose a relation into a certain normal form.
\end{enumerate}

\subsection{Algorithms for Testing}
\label{sec:algtest}
As we mentioned earlier we need algorithms for testing whether a decomposition 
is in 2NF, 3NF and BCNF, since a given schema may have many decompositions into a given normal form. Therefore,
simply computing one such decomposition with the \textit{Decomposition Algorithms} and comparing the student's solution to it is
not satisfactory. 

In this section we introduce algorithms for testing weather a decomposition 
is satisfying the lossless-join property, the
dependency preservation property, the 2NF, 3NF and BCNF properties. 
However, most of these algorithms depend on other, more general
algorithms, which we present first.  

\subsubsection{Attribute Closure}
Let $\alpha$  be a set of attributes. 
We call the set of attributes determined by $\alpha$ under a set $F$ of 
FDs the closure of $\alpha$ under $F$, denoted $\alpha\sp{+}$, thus \(\alpha \rightarrow \alpha\sp{+}\).
Figure~\ref{fig:attclosure} shows  pseudocode for an algorithm which can compute $\alpha\sp{+}$

\begin{figure}[htbp]
\hrule
\vspace{0.25cm}
\begin{alltt}
Algorithm Classical-AttributeClosure(\(\alpha\): set of attributes, 
              \(F\): set of FDs):
           return: set of attributes
/* This algorithm computes the closure of the set \(\alpha\) of
    attributes with respect to the set  \(F\) of FDs */
  \(\alpha\sp{+} = \alpha\)
  while(No more changes in \(\alpha\sp{+}\)) do
    foreach FD \(\beta \rightarrow \gamma\) in \(F\) do
      if \(\beta \subseteq \alpha\sp{+}\) then 
        \(\alpha\sp{+} \cup \gamma\)
      end if
    end foreach
  end while
  return \(\alpha\sp{+}\)
\end{alltt} 
\caption{Pseudocode for Algorithm Classical-AttributeClosure}\label{fig:attclosure}
\hrule
\end{figure}

Computing attribute closure is a very important part of every other normalization algorithm 
in LDBN. Therefore every improvement 
of the algorithm is significant. 
The \textit{Classical-AttributeClosure} algorithm is described in many textbooks including~\cite{bdb1, bdb2, bdb4}.
It has worst case behavior quadratic in the size of $F$ and
it is not suitable for large application as LDBN. We presented it only because it is
easy to follow. In our implementation of the learning environment we use a linear but more complicated 
algorithm for computing $\alpha\sp{+}$ 
called $SL_{FD}$-$Closure$~\cite{p10}, which is shown in Figure~\ref{alg:slfdclosure}

\begin{figure}[htbp]
\hrule
\vspace{0.25cm}
\begin{alltt}
Algorithm SLFD-Closure(\(\alpha\): set of attributes, 
              \(F\): set of FDs):
           return: set of attributes
/* This is a linear algorithm for computing the closure of 
  the set \(\alpha\) of attributes with respect to the set \(F\) of FDs */
  \(\alpha\sb{new} = \alpha\)
  \(\alpha\sb{old} = \alpha\)
  repeat
    foreach FD \(\beta \rightarrow \gamma\) in \(F\) do
      if \(\beta \subseteq \alpha\sb{new}\) then 
        \(\alpha\sb{new} = \alpha\sb{new} \cup \gamma\)
        \(F = F - \{\beta \rightarrow \gamma\}\) 
      elsif \(\gamma \subseteq \alpha\sb{new}\) then
        \(F = F - \{\beta \rightarrow \gamma\}\) 
      else
        \(F = F - \{\beta \rightarrow \gamma\}\) 
        \(F = F \cup \{\beta-\alpha\sb{new} \rightarrow \gamma-\alpha\sb{new}\}\) 
      end if
    end foreach 
  until ((\(\alpha\sb{new} = \alpha\sb{old}\)) or \(|F| = 0\))
  return \(\alpha\sp{+} = \alpha\sb{new}\)
\end{alltt} 
\caption{Pseudocode for Algorithm SLFD-Closure}\label{alg:slfdclosure}
\hrule
\end{figure}

The great efficiency improvement of such linear algorithms for computing the attribute closure
comes from the idea of adding the right-hand side of each FD once we have checked that
all their attributes are in the temporal closure. In
this way, the algorithms traverse the set of FDs
only once~\cite{p10}.

There are several direct uses of attribute closure~\cite[Section 7.4]{bdb4}:
\begin{description}
    \item[Superkey Test:] To test whether $\alpha$ is a superkey, we compute $\alpha\sp{+}$, and check whether $\alpha\sp{+}$ contains all attributes of $R$.
    \item[Computing $F\sp{+}$:] For each $\gamma \subseteq R$ we find the closure $\gamma\sp{+}$, and for each $S \subseteq \gamma\sp{+}$, we output an FD $\gamma \rightarrow S$.
\end{description}

\subsubsection{FD Test}
Figure~\ref{fig:fdtest} shows pseudocode for an algorithm for checking 
whether a functional dependency $\alpha \rightarrow \beta $ 
holds (i.e. is in $F\sp{+}$). In order to compute this we just need to check 
whether $\beta \subseteq \alpha\sp{+}$ holds~\cite[Section 7.4]{bdb4}.
  
\begin{figure}[htbp]
\hrule
\vspace{0.25cm}
\begin{alltt}
Algorithm FDTest(\(f\): FD, \(F\): set of FDs): return: boolean
/*This is a polynomial-time algorithm for checking whether \(f \in F\sp{+}\) holds*/
  \(\alpha\sp{+}\) = SLFD-Closure(\(\alpha, F\))   
  if \(\beta \in \alpha\sp{+}\) then
    return true
  else
    return false
  end if
\end{alltt}
\caption{Pseudocode for Algorithm FDTest}\label{fig:fdtest}
\hrule
\end{figure}

\subsubsection{Equivalence}
To determine whether two set of FDs $F$ and $G$ are equivalent, we need to prove that $F\sp{+} = G\sp{+}$. 
However, computing $F\sp{+}$ or
$G\sp{+}$ is exponential, therefore this approach cannot be recommended for practical use.
Fortunately, there is a much faster algorithm. We can conclude that $F$ and $G$ are
equivalent, if we can prove that all FDs in $F$ can be inferred from the set of FDs in $G$ and vice
versa. To achieve this we use the \textit{FDTest} algorithm.

\begin{figure}[htbp]
\hrule
\vspace{0.25cm}
\begin{alltt}
Algorithm Equivalence(\(F\): set of FDs, \(G\): set of FDs):
            return: boolean
/* This is a polynomial-time algorithm for checking whether \(F\sp{+} = G\sp{+}\), 
  i.e., whether the two sets of FDs are equivalent */ 
  foreach FD \(f : \alpha \rightarrow \beta\) in \(F\) do
    if not FDTest(\(f, G\)) then
      return false
  end foreach
  
  foreach FD \(g : \alpha \rightarrow \beta\) in \(G\) do
    if not FDTest(\(g, F\)) then
      return false
  end foreach
  return true
\end{alltt}
\caption{Pseudocode for Algorithm Equivalence}\label{alg:equivalence}
\hrule
\end{figure}

\subsubsection{Test Lossless-Join}
In Section~\ref{sec:decofrel} it was shown that a decomposition of $R$ into $R_1$ and $R_2$ 
has a lossless-join if one of the following FDs is in $F\sp{+}$
\begin{itemize}
  \item $f : R_1 \cap R_2 \rightarrow R_1$ 
  \item $g : R_1 \cap R_2 \rightarrow R_2$ 
\end{itemize}

However, the above test is not suitable for decompositions with more than two relation. 
Figure~\ref{alg:lossless} describes an algorithm which can be used to test
the lossless-join property of decompositions with more than two relations. The algorithm comes 
from~\cite[Section 3]{AhoBU79}, it is also described in \cite[Algorithm 11.1]{bdb1}.

It should be noted that there are two types of decompositions, acyclic and cyclic.
Roughly speaking, with acyclicity, the lossless-join property may be
checked pairwise; that is, two relations at a time.  With cyclic
decompositions, this is not possible.  
%%Most decompositions of
%%practical interest, and virtually all decompositions which are studied
%%in introductory database courses, are acyclic.  
The differences between acyclic and cyclic decompositions will
not be elaborated here; for an example of a cyclic decomposition see
\cite[Section 7]{AhoBU79}, while for a systematic presentation see
\cite{BeeriFMY83}. The testing algorithm which is presented here,
and which is used in the system, provides the correct
result for both acyclic and cyclic decompositions. 

%It may report failure for a
%cyclic decomposition, even if that decomposition is lossless.  It may
%furthermore fail if degenerate FDs of the form
% $\emptyset \rightarrow B$ are allowed; again, such dependencies are
%rarely if ever considered in introductory courses. In short, neither
%of these restrictions is significant from a pedagogical point of view.

%%%%%%%%%%%%%%%%%%%%%%%%%%%%%%%%%%%%%%%%%%%%%%%%%%%%%%%%%%%%%%%%%%%%%%%%%%%%%%%%
% This is not needed in the Elmasri's algorithm                                %
%%%%%%%%%%%%%%%%%%%%%%%%%%%%%%%%%%%%%%%%%%%%%%%%%%%%%%%%%%%%%%%%%%%%%%%%%%%%%%%%
%Acyclic decompositions admit so-called monotone join expressions
%\cite[Theorem 3.4]{BeeriFMY83}. This implies in particular that if
%$D$ is a decomposition of $R$, then there is an ordering
% $\langle R_1,R_2,..,R_n \rangle$
%of the elements of $D$ with the property that for each $i$,
% $1{}\leq{}i{}\leq{}n$, the following holds.
% \[
%   \left( \bigcup_{j=1}^{i} R_j \right) \cup R_{i+1} \neq \emptyset
% \]
% Call such a sequence \emph{monotone}.

\begin{figure}[htbp]
\hrule
\vspace{0.25cm}
\begin{alltt}
Algorithm TestLosslessJoin(\(R\): relation,
              \(F\): set of FDs which hold in \(R\),
              \(\mathcal{R}\): set of relations representing a decomposition of \(R\)):
        return: boolean
/* This algorithm tests whether the decomposition 
\(\mathcal{R}\) has the lossless-join property */
  
  Create an initial matrix \(S\) with one row \(i\) for each relation 
  \(R\sb{i} \in \mathcal{R}\), and one column \(j\) for each attribute \(A\sb{j}\) in R.  
  
   
  \(S(i, j) = b\sb{ij}\) for all matrix entries. 
  /* Each \(b\sb{ij}\) is a distinct symbol associated with indexes (i,j)*/
  
  foreach row \(i\) representing relation schema \(R\sb{i}\) do
    foreach column \(j\) representing attributes \(A\sb{j}\) do
      if \(A\sb{j} \in R\sb{i}\) then
        \(S(i, j) = a\sb{j}\) /* \(a\sb{j}\) is a distinct symbol associated with a index \(j\) */
      end if
    end foreach
  end foreach
  
  repeat
    foreach FD \(\alpha \rightarrow \beta\) in \(F\) do
      foreach row \(S\sb{i} \in S\) representing relation schema \(R\sb{S\sb{i}}\) do
        if \(\alpha \subseteq R\sb{S\sb{i}}\) then
          make the symbols in each column that correspond to
          an attribute in \(\beta\) be the same in all these rows as follows:
          if any of the rows has an \(a\) symbol for the column,
          set the other rows to the same \(a\) symbol in the column.
          If no \(a\) symbol exists for the attribute in any of the
          rows, choose one of the \(b\) symbols that appear in one
          of the rows for the attribute and set the other rows to
          that same \(b\) symbol in the column
        end if
      end foreach
    end foreach
  until \(S\) does not change
                  
If a row is made up entirely of \(a\) symbols, then the decomposition 
has the lossless join property; otherwise it does not.
\end{alltt}
\caption{Pseudocode for Algorithm TestLosslessJoin}\label{alg:lossless}
\hrule
\end{figure}

\vspace{1.5cm}

Here follows a short example which illustrates the \textit{TestLosslessJoin} algorithm. 
The example comes from~\cite[Figure 11.1]{bdb1}:

\vspace{0.2cm}

\indent \begin{tabular}[h]{l l}
  $R = \{A, B, C, D, E, F\}$ & $F = \{A \rightarrow B, C \rightarrow DE, AC \rightarrow F\}$ \\
  Decompose $R$ in: & $R_1 = \{B, E\}$ $R_2 = \{A, C, D, E, F\}$\\
\end{tabular}

\vspace{0.2cm}

\footnotesize
\indent \begin{tabular}[h]{c|c|c|c|c|c|c|c}
  \cline{2-7}
  & A & B & C & D & E & F \\
  \cline{2-7}
  \cline{2-7}
  $R_1$ & $b_{11}$ & $a_2$ & $b_{13}$ & $b_{14}$ & $a_3$ & $b_{16}$ \\
  $R_2$ & $a_1$ & $b_{22}$ & $a_{3}$ & $a_4$ & $a_5$ & $a_6$ \\
  \cline{2-7}
  \multicolumn{8}{l}{No changes to matrix after applying the FDs}
\end{tabular}
\normalsize
%\noindent No changes to matrix after applying the FDs, thus decomposition is not lossless-join.

\vspace{0.2cm}

The above matrix does not have a row only with $a$ symbols, thus the decomposition is 
not lossless. Let us consider a different decomposition of the same relation, which has
the lossless-join property:

\vspace{0.2cm}

\begin{tabular}[h]{l l}
  $R = \{A, B, C, D, E, F\}$ & $F = \{A \rightarrow B, C \rightarrow DE, AC \rightarrow F\}$ \\
  Decompose $R$ in: & $R_1 = \{A, B\}$ $R_2 = \{C, D, E\}$ $R_3 = \{A, C, F\}$\\
\end{tabular} 

\vspace{0.2cm}

\footnotesize
\hspace*{-0.5cm}\begin{minipage}[t]{0.5\linewidth}\centering
\begin{tabular}[h]{c|c|c|c|c|c|c|}
  \cline{2-7}
  & A & B & C & D & E & F \\
  \cline{2-7}
  \cline{2-7}
  $R_1$ & $a_{1}$ & $a_2$ & $b_{13}$ & $b_{14}$ & $b_{15}$ & $b_{16}$ \\
  $R_2$ & $b_{21}$ & $b_{22}$ & $a_{3}$ & $a_4$ & $a_5$ & $b_{25}$ \\
  $R_3$ & $a_1$ & $b_{32}$ & $a_{3}$ & $b_{34}$ & $b_{35}$ & $a_6$ \\
  \cline{2-7}
  \multicolumn{7}{l}{Matrix S at the beginning}
\end{tabular}
\end{minipage}
\hspace*{-0.5cm}
\begin{minipage}[t]{0.5\linewidth}\centering
\begin{tabular}[h]{c|c|c|c|c|c|c|c}
  \cline{2-7}
  & A & B & C & D & E & F \\
  \cline{2-7}
  \cline{2-7}
  $R_1$ & $a_{1}$ & $a_2$ & $b_{13}$ & $b_{14}$ & $b_{15}$ & $b_{16}$ \\
  $R_2$ & $b_{21}$ & $b_{22}$ & $a_{3}$ & $a_4$ & $a_5$ & $b_{25}$ \\
  $R_3$ & $a_1$ & $\cancel{b_{32}}$ $a_2$ & $a_{3}$ & $\cancel{b_{34}}$ $a_4$ &$\cancel{b_{35}}$ $a_5$ & $a_6$ \\
  \cline{2-7}
  \multicolumn{8}{l}{Matrix S after applying the first two FDs} \\
\end{tabular}
\end{minipage}
\normalsize
%\hrule
%\end{figure}

\subsubsection{Test Dependency Preservation} 
Using the \textit{Equivalence} algorithm testing whether a decomposition 
$R_1,...,R_n$ of $R$ is dependency preserving
becomes quite an easy task. We just need to compute: 
\[
  Equivalence(F, (F_1 \cup ... \cup F_n)) \mbox{ where } F_i \mbox{ is the set of FDs which holds in } R_i \mbox{ for each } i 
\] 

Figure~\ref{fig:dptest} shows pseudocode for the algorithm.

\begin{figure}[htbp]
\hrule
\vspace{0.25cm}
\begin{alltt}
Algorithm TestDependencyPreservation(\(F\): set of FDs which hold in \(R\),
           \((F\sb{1},...,F\sb{n})\): list of sets of FDs which holds in \(R\sb{i}\) for each \(i\)):
          return: boolean
/* This is a polynomial-time algorithm for testing the 
dependency-preservation property of a given decomposition */
  for each \(F\sb{i}\) in \((F\sb{1},...,F\sb{n})\) do
    \(G = G \cup F\sb{i}\)
  end foreach
  return Equivalence(G,F)
\end{alltt}
\caption{Pseudocode for Algorithm TestDependencyPreservation}\label{fig:dptest}
\hrule
\end{figure}

\subsubsection{Find All Candidate Keys}
The problem of finding all candidate keys is known to be NP-complete~\cite{p3}. 
Our learning environment uses a trivial algorithm for finding all candidate keys,
which is based on the \textit{SLFD-Closure} algorithm. The algorithm tests every possible
subset of attributes of the relation for being a superkey. If a subset of the superkey
is found which is also a superkey, then the first superkey is replace with the new one.  
Pseudocode for the algorithm is shown in Figure~\ref{alg:findkeys}. 

\begin{figure}[htbp]
\hrule
\vspace{0.25cm}
\begin{alltt}
Algorithm FindAllCandidateKeys(\(R\): relation,
              \(F\): set of FDs which hold in \(R\)):
            return: set of keys
/* This an exponential-time algorithm finds all candidate keys for \(R\)*/             
  \(\mathcal{K} = \emptyset\)
  foreach subset \(\gamma\) of \(R\) do
    \(\gamma\sp{+}\) = SLFD-Closure(\(\gamma, F\))
    if \(\gamma\sp{+} \subseteq R\) then
      remove all \(\kappa \in \mathcal{K}\) with  \(\gamma \subseteq \kappa\)
      if \(\gamma \nsubseteq \kappa\) for all \(\kappa \in \mathcal{K}\) then
        \(\mathcal{K} = \mathcal{K} \cup \gamma\)
      end if
    end if
  end foreach
  return \(\mathcal{K}\)
\end{alltt}
\caption{Pseudocode for Algorithm FindAllCandidateKeys}\label{alg:findkeys}
\hrule
\end{figure}

\subsubsection{Reduction By Resolution}
Often it is required to compute a cover for the projection of $F$ on a subschema $X$ of $R$, in 
other words, to find find the FDs on $R$ which are induced by $F$, thus:
\[ 
  F\sb{X}\sp{+} = \{\alpha\rightarrow\beta | \alpha \rightarrow \beta \in F\sp{+} \wedge \alpha \subseteq R \wedge \beta \subseteq R\}
\]
Although the problem of
finding such embedded cover of FDs is
known to be inherently exponential~\cite{p11}, it plays an important part of many of the algorithms
used by LDBN. The traditional approach is to compute $F\sp{+}$ and then project $F\sp{+}$ over the subschema $X$,
however, the produced cover
is always unnecessarily large~\cite{p4}. Still, for this difficult problem a simple 
algorithm called \textit{Reduction by Resolution}
was found by Gottlob~\cite{p4}. The algorithm is much more practical than the standard approach,
as it runs in polynomial time in a large number of cases, unlike the standard approach
which is exponential for each input. 

\begin{figure}[htbp]
\hrule
\vspace{0.25cm}
\begin{alltt}
Algorithm ReductionByResolution(\(R\): a relation,
                  \(F\) : set of FDs which hold in \(R\),
                  \(X\) : subset of attributes of \(R\)):
             return: set of FDs
/* This algorithm computes an embedded cover of FDs for \(F\sb{X}\sp{+}\) */
  \(G = F\)
  \(K = X - R\)
  while \(K \neq \emptyset\) do
    choose an element \(A \in K\),
    \(K = K - A\),
    \(RES = \emptyset\)
    foreach FD \(f\) in \(F\) of the form \(Y \rightarrow A\) do
      foreach FD \(g\) in \(G\) of the form \(AZ \rightarrow B\) do
        \(h = YZ \rightarrow B\),
        /* h is the resolvent of f and g */
        if h not trivial then \(RES = RES \cup \{h\}\)
      end foreach
    end foreach
    foreach FD \(f\) in \(G\) do
      if \(A\) occurs in \(f\) then \(G = G - \{f\}\)
    end foreach
    \(G = G \cup RES\)
  end while
  return \(G\)
\end{alltt}
\caption{Pseudocode for Algorithm ReductionByResolution}\label{alg:rbr}
\hrule
\end{figure}

\subsubsection{Test BCNF}
To check whether a non-trivial dependency $\alpha \rightarrow \beta$  causes a violation of BCNF,
we must use the algorithm \textit{SLFD-Closure} to test whether $\alpha$ is a superkey. 
It suffices to check only the dependencies in the given set $F$ for violation of BCNF, 
rather than checking all dependencies in $F\sp{+}$~\cite{bdb4}.  
If none of the dependencies in $F$ causes a violation of BCNF, 
then none of the dependencies in $F\sp{+}$ will cause a violation of BCNF either.
However, using only $F$ is incorrect when testing a relation in a decomposition of $R$~\cite{bdb4}.
Let us consider the following example:

\begin{center}
\begin{tabular}[h]{l l}
  $R = \{A, B, C, D, E\}$ & $F = \{A \rightarrow B, BC \rightarrow D\}$ \\
  Decompose $R$ in:  & $R_1 = \{A, B\} \mbox{ and } R_2 = \{A, C, D, E\}$ \\ 
\end{tabular}
\end{center}

Neither of the dependencies in $F$ contain only attributes from
$(A,C,D,E)$ so we might be mislead into thinking $R_2$ satisfies BCNF. In fact, 
dependency $AC \rightarrow D$ in $F\sp{+}$ shows $R_2$ is not in BCNF. To avoid this we can
use the \textit{Reduction by Resolution} algorithm to compute covers for $R_1$ and $R_2$. 

The algorithm uses the \textit{Reduction by Resolution} algorithm which is known to be exponential
in some bad cases, thus the \textit{BCNF Test} is also exponential. In fact, the test is known to 
be co-NP-complete~\cite{p4}.

\subsubsection{Test 3NF}
The algorithm \textit{Test 3NF} is similar to the \textit{Test BCNF} one. 
By using the \textit{Reduction By Resolution} algorithm
we must first compute the
embedded FD cover $F_i$ for each relation $R_i$ in a decomposition $R_1,...,R_n$ of $R$. Then we use 
the \textit{SLFD-Closure} algorithm to test each FD in $F_i$ of the form $\alpha \rightarrow \beta$ 
whether $\alpha$ is a superkey in $R_i$. 
If the $\alpha$ is not a superkey, we have to verify whether each attribute in the $\beta$ 
is contained in a candidate key of $R$, i.e., it is prime. Testing attribute for being a prime
attribute is known to be an NP-complete problem~\cite{p3}. For this part of the \textit{Test 3NF} algorithm
we can use our \textit{Find All Candidate Keys} algorithm in order
to compute all the candidate keys in $R_i$ and then test whether the attributes in $\alpha$ are part of any 
of them.

\subsubsection{Test 2NF}
%Testing whether a a decomposition is in 2NF is even harder than the \textit{Test 3NF}. 
Once again we compute the
embedded FD cover for each relation $R_i$ in a decomposition $R_1,...,R_n$ of $R$. After that 
we compute the set of all candidate keys for each relation $R_i$, we refer to this set as $K_i$.
 
To test whether $R_i$ is in 2NF we have to ensure that each non-key attribute is fully
functional dependent on every candidate key in $K_i$. To simply the problem we can test
each FD in $F_i$ of the form $\alpha \rightarrow \beta$ whether $\alpha$ is a proper subset
of any key candidate in $K_i$. If this is true and there is a non-key attribute in $\beta$ then
$R_i$ is not in 2NF, otherwise the relation $R_i$ is in 2NF.

The problem of testing whether a relation is in 2NF is NP-complete, 
since it involves finding all candidate keys.

\subsection{Decomposition Algorithms}
\label{sec:algdec}
In previous sections we already decomposed some relations without
a formal definition of the rules which we were following during this process. 
In this section we present algorithms for finding a minimal cover of a set of FDs and 
for decomposing any proposed relation into 3NF and BCNF.  

\subsubsection{Find Minimal Cover}  
With the help of the \textit{SLFD-Closure} algorithm we can now define an algorithm 
for computing a minimal cover of a set \textit{F} of FDs. 
The algorithm comes form~\cite[Section 6.3.1]{bdb2}. First, it finds and eliminates 
redundant attributes in the left-hand side (LHS) and right-hand side (RHS) 
of each FD in \textit{F} by using the \textit{SLFD-Closure} algorithm. Then, it eliminates FDs with 
empty LHS, which could. Pseudocode for the algorithm is shown in Figure~\ref{alg:mincov}.

\begin{figure}[htbp]
\hrule
\vspace{0.25cm}
\begin{alltt}
Algorithm FindMinimalCover(\(F\): set of FDs):
             return: set of FDs
/* This algorithm computes a minimal cover of FDs of set \(F\) of FDs */
  \(F\sb{c} = F\)
  repeat
    foreach FD \(\alpha \rightarrow \beta\) in \(F\sb{c}\) do
      /* Find redundant attributes in the left-hand side of the FD */
      foreach \(A \in \alpha\) do
        /* Checks if A is redundant */
        if \(\beta \subseteq\) SLFD-Closure(\(\alpha-A, F\sb{c}\)) then
          replace \(\alpha \rightarrow \beta\) with \((\alpha-A) \rightarrow \beta\)
        end if
      end foreach
      /* Find redundant attributes in the right-hand side of the FD */
      foreach \(B \in \beta\) do
        /* Checks if B is redundant */
        if \(B \subseteq\) SLFD-Closure(\(\alpha, F\sb{c}-(\alpha \rightarrow \beta) \cup (\alpha \rightarrow (\beta - B))\)) then
          replace \(\alpha \rightarrow \beta\) with \((\alpha \rightarrow (\beta-B)\)
        end if
      end foreach
    end foreach
    
    Delete all FDs FD of the form \(\alpha \rightarrow \emptyset\) form \(F\sb{c}\)
      
    Using the \(Union Rule\) combine all FDs of the form \(\alpha \rightarrow \beta,...,\alpha \rightarrow \gamma\)
      into one FD \(\alpha \rightarrow (\beta \cup ... \cup \gamma)\)
      
  until \(F\sb{c}\) does not change
  return \(F\sb{c}\)
\end{alltt}
\caption{Pseudocode for Algorithm FindMinimalCover}\label{alg:mincov}
\hrule
\end{figure}

\subsubsection{Decomposition 3NF}
The next algorithm decomposes a relation schema in 3NF. The algorithm
ensures that each new relation schema is in 3NF, thus the decomposition is in 3NF. 
Furthermore, the decomposition is dependency preserving and lossless-join. 
The algorithm comes form~\cite[Section 6.8]{bdb2}. 
First, we construct new relations corresponding to each FD in in $F_{c}$ (the minimal cover of $F$). 
Then we need to eliminate relations, the attributes of
which are subset of the attributes of another relation. 
After that, we find all the candidate keys of
the initial relation, this is necessary in order to determine  whether one of them is present
in the new set of relations, which is required by the algorithm to ensure
dependency preservation. If this is not the case, the algorithm creates
a new relation \(\mathcal{R}\sb{i}\) = any candidate key for \(R\).
Figure~\ref{alg:dec3nf} shows pseudocode for the algorithm.

\begin{figure}[htbp]
\hrule
\vspace{0.25cm}
\begin{alltt}
Algorithm Decomposition3NF(\(R\): set of FDs,
             \(F\sb{c}\): minimal cover of a set of FDs which hold in \(R\)):
          return: set of relational schemas  
/* This algorithm decomposes a relation schema in 3NF */
  \(\mathcal{R} = \emptyset\)
  i = 0
  foreach FD \(\alpha \rightarrow \beta\) in \(F\sb{c}\) do
    i = i + 1
    create a relation schema \(\mathcal{R}\sb{i} = \alpha \cup \beta\)
    assign \(\mathcal{R}\sb{i}\) the FDs  \(F\sb{i} = \{\alpha\sp{'} \rightarrow \beta\sp{'} \in F\sb{c}\ | \alpha\sp{'} \cup \beta\sp{'} \subseteq \mathcal{R}\sb{i}\}\)
    \(\mathcal{R} = \mathcal{R} \cup \mathcal{R}\sb{i}\)
  end foreach
  
  if none of the schemas \(\mathcal{R}\sb{j}, 1\leq{}j\leq{}i\) contains a candidate key for \(R\) then
    i = i + 1
    create a relation schema \(\mathcal{R}\sb{i}\) = any candidate key for \(R\)
    \(\mathcal{R} = \mathcal{R} \cup \mathcal{R}\sb{i}\)
    
  Delete all relational schemas \(\mathcal{R}\sb{a}\) from \(\mathcal{R}\) with \(\mathcal{R}\sb{a} \subseteq \mathcal{R}\sb{j},  1 \leq j \leq i\)
  
  return \(\mathcal{R}\)
\end{alltt}
\caption{Pseudocode for Algorithm Decomposition3NF}\label{alg:dec3nf}
\hrule
\end{figure}

\subsubsection{Decomposition BCNF}
There is a simple algorithm for decomposing a relational schema into BCNF.
It is described in~\cite[Section 6.9]{bdb2}. The algorithm always 
procures a lossless-join decomposition, however, we must run the 
\textit{TestDependencyPreservation} algorithm on
the decomposition to check whether the decomposition is dependency preserving or not. 
As we mentioned in Section~\ref{sec:BCNF},
it is no always possible to find a dependency-preserving BCNF decomposition.
Figure~\ref{alg:decbcnf} shows pseudocode for the algorithm.

\begin{figure}[htbp]
\hrule
\vspace{0.25cm}
\begin{alltt}
Algorithm DecompositionBCNF(\(R\): a relation,
             \(F\): a set of FDs which hold in \(R\)):
          return: set of relational schemas
/* This algorithm decomposes a relation schema in BCNF */
  \(Z = R\)
  done = false
  while not done do
    if there is a schema \(R\sb{i}\) in \(Z\)  that is not in BCNF then begin
      let \(\alpha \rightarrow \beta\) be a nontrivial FD that holds on \(R\sb{i}\)
        such that \(\alpha \nrightarrow R\sb{i}\) /* \(\alpha\) is not a superkey */
        and  \(\alpha \cap \beta \neq \emptyset\) then  
          decompose \(R\sb{i}\) in \(R\sb{i1} = \alpha \cup \beta\) and \(R\sb{i2} = R\sb{i} - \beta\)
          remove \(R\sb{i}\) from \(Z\) 
          \(Z = Z \cup R\sb{i1} \cup R\sb{i2}\)
    else 
      done = true
    end if 
  end while
  return \(Z\)
\end{alltt}
\caption{Pseudocode for Algorithm DecompositionBCNF}
\label{alg:decbcnf}
\hrule
\end{figure}
