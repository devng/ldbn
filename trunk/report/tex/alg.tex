\section{Normalization Algorithms}
\label{sec:alg}
In this section we present different normalization algorithms which are used
in LDBN. A proof of their correctness and complexity would be out of the 
scope of this report and therefore it is omitted. We provide the
reader with textual description and/or pseudo code for each algorithm. 
A reference to additional information is also given. However, in this section we do not
illustrate how the algorithms are applied in the learning environment in order
to perform the \textit{Solve Assignment} and the \textit{Check Solution} functions.
This is done in Section~\ref{sec:keyfunctions}.

We can divide the normalization algorithms used by LDBN into two different groups:

\begin{enumerate}
  \item \textit{Algorithms for Testing} are used to test whether a relation is satisfying a certain normal form criteria.
  \item \textit{Decomposition Algorithms} are used to automatically decompose a relation into a certain normal form.
\end{enumerate}

\subsection{Algorithms for Testing}
\label{sec:algtest}
As we mention earlier we need algorithms for testing whether a decomposition 
is in 2NF, 3NF and BCNF, since a given schema may have many decompositions into a given normal form. Therefore,
simply computing one such decomposition with the \textit{Decomposition Algorithms} and comparing the student's solution to it is
not satisfactory. 

In this section we introduce algorithms for testing weather a decomposition 
is satisfying the lossless-join property, the
dependency preservation property, the 2NF, 3NF and BCNF properties. 
However, most of these algorithms depend on other, more general
algorithms, which we present first.  

\subsubsection{Attribute Closure}
Let $\alpha$  be a set of attributes. 
We call the set of attributes determined by $\alpha$ under a set $F$ of 
FDs the closure of $\alpha$ under $F$, denoted $\alpha +$. The following algorithm
computes $\alpha+$

\begin{alltt}
INPUT
  \(\alpha\)  : a set of attributes
  \(F\)  : a set of FDs
OUTPUT
  \(\alpha+\) : a complete set of attributes  \(\alpha+\)  with  \(\alpha \rightarrow \alpha+\)
BEGIN
  \(\alpha+ = \alpha\)
  while(No more changes in \(\alpha+\)) do
    foreach FD \(\beta \rightarrow \gamma\) in \(F\) do
      if \(\beta \subseteq \alpha+\) then \(\alpha+ \cup \gamma\)
    return \(\alpha+\)
\end{alltt} 

The \textit{Attribute Closure} algorithm is one of the most important algorithms used in LDBN. It is
part of every other normalization algorithm in the system. Therefore every improvement 
of the algorithm is significant. 
The above algorithm is described in many textbooks including~\cite{bdb1, bdb2, bdb4}.
It has worst case behavior quadratic in the size of $F$ and
it is not suitable for large application like LDBN. We presented it only because it is
easy to follow. In our learning environment we use a linear but more complicated algorithm for computing $\alpha+$ 
called $SL_{FD}-Closure$~\cite{p10}:

\begin{alltt}
INPUT
  \(\alpha\)  : a set of attributes
  \(F\)  : a set of FDs
OUTPUT
  \(\alpha+\) : a complete set of attributes  \(\alpha+\)  with  \(\alpha \rightarrow \alpha+\)
BEGIN
  \(\alpha\sb{new} = \alpha\)
  \(\alpha\sb{old} = \alpha\)
  repeat
    foreach FD \(\beta \rightarrow \gamma\) in \(F\) do
      if \(\beta \subseteq \alpha\sb{new}\) then 
        \(\alpha\sb{new} \cup \gamma\)
        \(F = F - \{\beta \rightarrow \gamma\}\) 
      elsif \(\gamma \subseteq \alpha\sb{new}\) then
        \(F = F - \{\beta \rightarrow \gamma\}\) 
      else
        \(F = F - \{\beta \rightarrow \gamma\}\) 
        \(F = F \cup \{\beta-\alpha\sb{new} \rightarrow \gamma-\alpha\sb{new}\}\) 
      end if
    end foreach 
  until ((\(\alpha\sb{new} = \alpha\sb{old}\)) or \(|F| = 0\))
  return \(\alpha+ = \alpha\sb{new}\)
\end{alltt} 

There are several direct uses of the \textit{Attribute Closure} algorithm~\cite{bdb4}:
\begin{description}
    \item[Superkey Test:] To test whether $\alpha$ is a superkey, we compute $\alpha+$, and check whether $\alpha+$ contains all attributes of $R$.
    \item[Computing F+:] For each $\gamma \subseteq R$ we find the closure $\gamma+$, and for each $S \subseteq \gamma+$, we output a FD $\gamma \rightarrow S$.
\end{description}

\subsubsection{FD Test}
To check whether a functional dependency $\alpha \rightarrow \beta $ 
holds (i.e. is in $F+$), just check whether $\beta \subseteq \alpha+$ holds~\cite{bdb4}.
 
\begin{alltt}
INPUT 
  \(f\) : a FD of the form \(\alpha \rightarrow \beta\)
  \(F\) : a set of FDs
OUTPUT 
  true if \(f \in F+\), false otherwise
BEGIN
  compute \(\alpha+ = AttributeClosure(\alpha, F)\)   
  if \(\beta \in \alpha+\) then
    return true
  else
    return false
\end{alltt}
  
\subsubsection{Equivalence}
To determine whether two set of FDs $F$ and $G$ are equivalent, we need to prove that $F+ = G+$. 
However, computing $F+$ or
$G+$ is exponential, therefore this approach cannot be recommended for practical use.
Fortunately, there is a much faster algorithm. We can conclude that $F$ and $G$ are
equivalent, if we can prove that all FDs in $F$ can be inferred from the set of FDs in $G$ and vice
versa. To achieve this we use the \textit{FD Test} algorithm.

\begin{alltt}
INPUT 
  \(F\) : a set of FDs
  \(G\) : a set of FDs
OUTPUT 
  true if \(F \equiv G\), false otherwise
BEGIN
  foreach FD \(f : \alpha \rightarrow \beta\) in \(F\) do
    if not \(FDTest(f, G\)) then
      return false
  end foreach
  
  foreach FD \(g : \alpha \rightarrow \beta\) in \(G\) do
    if not \(FDTest(g, F\)) then
      return false
  end foreach
  
  return true
\end{alltt}

\subsubsection{Lossless-Join Property Test}
In Section~\ref{sec:decofrel} it was shown that a decomposition of $R$ into $R_1$ and $R_2$ 
has a lossless-join if one of the following FDs is in $F+$
\begin{itemize}
  \item $f : R_1 \cap R_2 \rightarrow R_1$ 
  \item $g : R_1 \cap R_2 \rightarrow R_2$ 
\end{itemize}

We can use the \textit{FD Check} algorithm to test whether $f$ or $g$ are in $F+$. Furthermore,
in~\cite{bdb2}, Section~6.5.1, it has been shown inductively that the above rule can be applied for a 
decomposition with $n$ subschemas by using the associativity of the natural join~($\Join$):
$$
  R_1 \Join (R_2 \Join (...\Join(R_{n-1} \Join R_{n}) \newline
  =R_1 \Join (R_2 \Join (...\Join(R_{n-2} \Join R_{n-1}')  \newline
  =
  ...
  = R
$$
Knowing these facts, we can now define a simple algorithm for testing the lossless-join
property of a decomposition \(\mathcal{R}\):
   
\begin{alltt}
INPUT 
  \(R\) : a relation
  \(F\) : a set of FDs which hold in \(R\)
  \(\mathcal{R}\) : a set of relations representing a decomposition of \(R\).
OUTPUT 
  true if  \(\mathcal{R}\) is lossless-join, false otherwise
BEGIN
  K = first element of \(\mathcal{R}\)
  foreach \(R\sb{i} \in \mathcal{R}\) for \(i \geq 2\) do
    if \(FDCheck(K \cap R\sb{i} \rightarrow K, F)\) or \(FDCheck(K \cap R\sb{i} \rightarrow R\sb{i}, F)\) then
      \(K = K \cup R\sb{i}\)
    else
      return false
    end if
  end foreach
  return true
\end{alltt}

%In the above algorithm we are using the $\cup$ operator instead of $\Join$. This is
%permitted, since we are using only the attributes of the relation $R$ and not actual instances (i.e.,
%relation with its contents).

\subsubsection{Dependency Preservation Test} 
Using the \textit{Equivalence} algorithm testing whether a decomposition $R_1,...,R_n$ of $R$ is dependency preserving
becomes quite an easy task. We just need to compute: $$Equivalence(F, (F_1 \cup ... \cup F_n)) \mbox{ where } F_i \mbox{ is the set of FDs which holds in } R_i$$ 

\subsubsection{Find All Candidate Keys}
The problem of finding all candidate keys is know to be NP-complete~\cite{p3}. 
Our leaning environment uses an trivial algorithm for finding all the candidate keys,
which is based on the \textit{Attribute Closure} algorithm. The algorithm test every possible
subset of attributes of the relation for being a superkey. If a subset of the superkey
is found which is also a superkey, then the first superkey is replace with the new one.   

\begin{alltt}
INPUT 
  \(R\) : a relation
  \(F\) : a set of FDs which hold in \(R\)
OUTPUT 
  \(\mathcal{K}\) : a set of all candidate keys for \(R\)
BEGIN
  \(\mathcal{K} = \emptyset\)
  foreach subset \(\gamma\) of \(R\) do
    compute \(\gamma+\)
    if \(\gamma+ \) contains all attributes of \(R\) then
      remove all \(\kappa \in \mathcal{K}\) with  \(\gamma \subseteq \kappa\)
      if \(\gamma \nsubseteq \kappa\) for all \(\kappa \in \mathcal{K}\) then
        \(\mathcal{K} \cup \gamma\)
      end if
    end if
  end foreach
return \(\mathcal{K}\)
\end{alltt}

\subsubsection{Reduction By Resolution}
Often is required to compute a cover for the projection of $F$ on a subschema $X$ of $R$, in 
other words, to find find the FDs on $R$ which are induced by $F$. Altought the problem of
finding such embedded cover of FDs is
known to be inherently exponential~\cite{p11}, it plays an important part of many of the algorithms
used by LDBN. The traditional approach is to compute $F+$ and then project $F+$ over the subschema $X$,
however, the produced cover
is always unnecessarily large~\cite{p4}. Still, for this difficult problem a simple 
algorithm called \textit{Reduction by Resolution}
was found by Gottlob~\cite{p4}. The algorithm is much more practical than the standard approach,
as it runs in polynomial time in a large number of cases, unlike the standard approach
which is exponential for each input. 

\begin{alltt}
INPUT 
  \(R\) : a relation
  \(F\) : a set of FDs which hold in \(R\)
  \(X\) : a subset of attributes of \(R\)
OUTPUT 
  \(F\sb{X}\) : a cover for \(F\sb{X}+\) with 
           \(F\sb{X}+ = \{\alpha\rightarrow\beta | \alpha \rightarrow \beta \in F+ \wedge \alpha \subseteq R \wedge \beta \subseteq R\}\)
BEGIN 
  \(G = F\)
  \(K = X - R\)
  while \(K \neq \emptyset\) do
    choose an element \(A \in K\),
    \(K = K - A\),
    \(RES = \emptyset\)
    foreach FD \(f\) in \(F\) of the form \(Y \rightarrow A\) do
      foreach FD \(g\) in \(G\) of the form \(AZ \rightarrow B\) do
        \(h = YZ \rightarrow B\),
        /* h is the resolvent of f and g */
        if h not trivial then \(RES = RES \cup \{h\}\)
      end foreach
    end foreach
    foreach FD \(f\) in \(G\) do
      if \(A\) occurs in \(f\) then \(G = G - \{f\}\)
    end foreach
    \(G = G \cup RES\)
  end while
  return \(G\)
\end{alltt}

\subsubsection{BCNF Test}
To check whether a non-trivial dependency $\alpha \rightarrow \beta$  causes a violation of BCNF,
we must use the \textit{Attribute Closure} to test whether $\alpha$ is a superkey. 
It suffices to check only the dependencies in the given set $F$ for violation of BCNF, 
rather than checking all dependencies in $F+$~\cite{bdb4}.  
If none of the dependencies in $F$ causes a violation of BCNF, 
then none of the dependencies in $F+$ will cause a violation of BCNF either.
However, using only $F$ is incorrect when testing a relation in a decomposition of $R$~\cite{bdb4}.
Let us consider the following example:

\begin{center}
\begin{tabular}[h]{l l}
  $R = \{A, B, C, D, E\}$ & $F = \{A \rightarrow B, BC \rightarrow D\}$ \\
  Decompose $R$ in:  & $R_1 = \{A, B\} \mbox{ and } R_2 = \{A, C, D, E\}$ \\ 
\end{tabular}
\end{center}

Neither of the dependencies in $F$ contain only attributes from
$(A,C,D,E)$ so we might be mislead into thinking $R_2$ satisfies BCNF. In fact, 
dependency $AC \rightarrow D$ in $F+$ shows $R_2$ is not in BCNF. To avoid this we can
use the \textit{Reduction by Resolution} algorithm to compute covers for $R_1$ and $R_2$. 

The algorithm uses the \textit{Reduction by Resolution} algorithm which is know to be exponential
in some bad cases, thus the \textit{BCNF Test} is also exponential. In fact, the test is know to 
be co-NP-complete~\cite{p4}.

\subsubsection{3NF Test}
The \textit{3NF Test} is similar to the \textit{BCNF Test}. By using the \textit{Reduction by Resolution} algorithm
we must first compute the
embedded FD cover $F_i$ for each relation $R_i$ in a decomposition $R_1,...,R_n$ of $R$. Then we use 
the \textit{Attribute Closure} to test each FD in $F_i$ of the form $\alpha \rightarrow \beta$ 
whether $\alpha$ is a superkey in $R_i$. 
If the $\alpha$ is not a superkey, we have to verify whether each attribute in the $\beta$ 
is contained in a candidate key of $R$, i.e., it is prime. Testing attribute for being a prime
attribute is know to be a NP-complete problem~\cite{p3}. For this part of the \textit{3NF Test} algorithm
we can use our \textit{Find All Candidate Keys} algorithm in order
to compute all the candidate keys in $R_i$ and then test whether the attributes in $\alpha$ are part of any 
of them.

\subsubsection{2NF Test}
%Testing whether a a decomposition is in 2NF is even harder than the \textit{3NF Test}. 
Once again we compute the
embedded FD cover for each relation $R_i$ in a decomposition $R_1,...,R_n$ of $R$. After that 
we compute each candidate key for each for each a relation $R_i$, we refer to this set as $K_i$.
 
To test whether $R_i$ is in 2NF we have to ensure that each non-key attribute is fully
functional dependent on every candidate key in $K_i$. To simply the problem we can test
each FD in $F_i$ of the form $\alpha \rightarrow \beta$ whether $\alpha$ is a proper subset
of a key candidate in $K_i$. If this is true and the attributes in $\beta$ are not prime then
$R_i$ is not in 2NF.

The \textit{2NF Test} is NP-complete, since it involves finding all candidate keys.

\subsection{Decomposition Algorithms}
\label{sec:algdec}
In previous sections we already decomposed some relations without
a formal definition of the rules which we were following during this process. 
In this section we present algorithms for finding a minimal cover of a set of FDs and 
for decomposing any proposed relation into 3NF and BCNF.  

\subsubsection{Find Minimal Cover}
With the help of the \textit{Attribute Closure} algorithm we can now define an algorithm 
for computing a minimal (canonical) cover of a set $F$ of FDs. The algorithm comes form~\cite{bdb2}, Section~6.3.1.

\begin{alltt}
INPUT 
  \(F\)  : a set of FDs
OUTPUT 
  \(F\sb{c}\) : a minimal (canonical) cover of \(F\)
BEGIN

\(F\sb{c} = F\)
repeat
  /* Find redundant attributes in the left-hand side of each FD */
  foreach FD \(\alpha \rightarrow \beta\) in \(F\sb{c}\) do
    foreach \(A \in \alpha\) do
      /* Checks if A is redundant */
      if \(\beta \subseteq AttributeClosure(F\sb{c}, \alpha-A)\) then
        replace \(\alpha \rightarrow \beta\) with \((\alpha-A) \rightarrow \beta\)
      end if
    end foreach
  end foreach
    
  /* Find redundant attributes in the right-hand side of each FD */
  foreach FD \(\alpha \rightarrow \beta\) in \(F\sb{c}\) do
    foreach \(B \in \beta\) do
      /* Checks if B is redundant */
      if \(B \subseteq AttributeClosure(F\sb{c}-(\alpha \rightarrow \beta) \cup (\alpha \rightarrow (\beta - B)), \alpha)\) then
        replace \(\alpha \rightarrow \beta\) with \((\alpha \rightarrow (\beta-B)\)
      end if
    end foreach
  end foreach
    
  Delete all FDs FD of the form \(\alpha \rightarrow \emptyset\) form \(F\sb{c}\)
    
  Using the \(Union Rule\) combine all FDs of the form \(\alpha \rightarrow \beta,...,\alpha \rightarrow \gamma\)
    into one FD \(\alpha \rightarrow (\beta \cup ... \cup \gamma)\)
    
until \(F\sb{c}\) does not change
return \(F\sb{c}\)
\end{alltt}

For the sake of simplicity we iterate over all FDs in $F$ several times, in the actual implementation
the number of iteration could be reduced. 

\subsubsection{3NF Decomposition}
The following algorithm decomposes a relation schema in 3NF. The algorithm
ensures that each new relation schema is in 3NF, thus the decomposition is in 3NF. 
Furthermore, the decomposition is dependency preserving and lossless-join. 
The algorithm comes form~\cite{bdb2}, Section~6.8. 
First, we build new relations corresponding to each FD in in $F_{c}$ (the minimal cover of $F$). 
Then we need to eliminate relations, the attributes of
which are subset of the attributes of another relation. 
After that, we find all the candidate keys of
the initial relation, this is necessary in order to determine  whether one of them is present
in the new set of relations, which is required by the algorithm to ensure
dependency preservation. If this is not the case, the algorithm creates
a new relation \(\mathcal{R}\sb{i}\) = any candidate key for \(R\).
Here follows a pseudo code of the algorithm:

\begin{alltt}
INPUT 
  \(R\)  : a relation
  \(F\sb{c}\) : a canonical cover of a set of FDs which hold in \(R\)
OUTPUT 
  \(\mathcal{R}\) : a set of relational schemas representing a decomposition in 3NF 
BEGIN
  \(\mathcal{R} = \emptyset\)
  i = 0
  foreach FD \(\alpha \rightarrow \beta\) in \(F\sb{c}\) do
    i = i + 1
    create a relation schema \(\mathcal{R}\sb{i} = \alpha \cup \beta\)
    assign \(\mathcal{R}\sb{i}\) the FDs  \(F\sb{i} = \{\alpha\sp{'} \rightarrow \beta\sp{'} \in F\sb{c}\ | \alpha\sp{'} \cup \beta\sp{'} \subseteq \mathcal{R}\sb{i}\}\)
    \(\mathcal{R} = \mathcal{R} \cup \mathcal{R}\sb{i}\)
  end foreach
  
  if none of the schemas \(\mathcal{R}\sb{j}, 1 \leq j \leq i\) contains a candidate key for \(R\) then
    i = i + 1
    create a relation schema \(\mathcal{R}\sb{i}\) = any candidate key for \(R\)
    \(\mathcal{R} = \mathcal{R} \cup \mathcal{R}\sb{i}\)
    
  Delete all relational schemas \(\mathcal{R}\sb{a}\) from \(\mathcal{R}\) with \(\mathcal{R}\sb{a} \subseteq \mathcal{R}\sb{j},  1 \leq j \leq i\)
  
  return \(\mathcal{R}\)
\end{alltt}

\subsubsection{BCNF Decomposition}
There is a simple algorithm for decomposing a relational schema into BCNF.
It is described in~\cite{bdb2}, Section~6.9:

\begin{alltt}
INPUT
  \(R\) : a relation
  \(F\) : a set of FDs which hold in \(R\)
OUTPUT
  \(Z\) : a set of relational schemas representing a decomposition in BCNF 
BEGIN
\(Z = R\)
done = false

while not done do
  if there is a schema \(R\sb{i}\) in \(Z\)  that is not in BCNF then begin
    let \(\alpha \rightarrow \beta\) be a nontrivial FD that holds on \(R\sb{i}\)
      such that \(\alpha \nrightarrow R\sb{i}\) /* \(\alpha\) is not a superkey */
      and  \(\alpha \cap \beta \neq \emptyset\) then  
        decompose \(R\sb{i}\) in \(R\sb{i1} = \alpha \cup \beta\) and \(R\sb{i2} = R\sb{i} - \beta\)
        remove \(R\sb{i}\) from \(Z\) 
        \(Z = Z \cup R\sb{i1} \cup R\sb{i2}\)
  else 
    done = true
  end if 
return \(Z\)
\end{alltt}

The algorithm always 
procures a lossless-join decomposition, however, we must run the \textit{Dependency Preservation Test} on
the decomposition to check whether the decomposition is dependency preserving or not. As we mention in Section~\ref{sec:BCNF},
it is no always possible to find a dependency preserving BCNF decomposition.

